%% LyX 1.1 created this file.  For more info, see http://www.lyx.org/.
%% Do not edit unless you really know what you are doing.
\documentclass[english]{article}
\usepackage[OT1]{fontenc}
\usepackage[latin1]{inputenc}
\usepackage{geometry}
\geometry{verbose,letterpaper,tmargin=1in,bmargin=1in,lmargin=1in,rmargin=1in}
\usepackage{babel}
\usepackage{verbatim}

\makeatletter

%%%%%%%%%%%%%%%%%%%%%%%%%%%%%% LyX specific LaTeX commands.
\providecommand{\LyX}{L\kern-.1667em\lower.25em\hbox{Y}\kern-.125emX\@}

\makeatother
\begin{document}

\section{Introduction}

SciPy is a collection of mathematical algorithms and convenience functions
built on the Numeric extension for Python. It adds significant power
to the interactive Python session by exposing the user to high-level
commands and classes for the manipulation and visualization of data.
With SciPy, an interactive Python session becomes a data-processing
and system-prototyping environment rivaling sytems such as Matlab,
IDL, Octave, R-Lab, and SciLab. 

The additional power of using SciPy within Python, however, is that
a powerful programming language is also available for use in developing
sophisticated programs and specialized applications. Scientific applications
written in SciPy benefit from the development of additional modules
in numerous niche's of the software landscape by developers across
the world. Everything from parallel programming to web- and data-base
subroutines and classes have been made available to the Python programmer.
All of this power is available in addition to the more mathematical
libraries in SciPy.

This document provides a tutorial for the first-time user of SciPy
to help get started with some of the features available in this powerful
package. It is assumed that the user has already installed the package.
Some general Python facility is also assumed such as could be acquired
by working through the Tutorial in the Python distribution. 


\section{General help}

Python provides the facility of documentation strings. The functions
and classes available in SciPy use this method for on-line documentation.
There are two methods for reading these messages and getting help.
Python provides the command help in the pydoc module. Entering this
command with no arguments (i.e. \textgreater{}\textgreater{}\textgreater{}
help ) launches an interactive help session that allows searching
through the keywords and modules available to all of Python. Running
the command help with an object as the argument displays the calling
signature, and the documentation string of the object.

The pydoc method of help is sophisticated but uses a pager to display
the text. Sometimes this can interfere with the terminal you are running
the interactive session within (like emacs). A scipy-specific help
system is also available under the command scipy.help. The signature
and documntation string for the object passed to the help command
are printed to standard output (or to a writeable object passed as
the third argument). The second keyword argument of {}``scipy.help{}''
defines the maximum width of the line for printing.

If a module is passed as the argument to help than a list of the functions
and classes defined in that module is printed. 

\textbf{Example:}

\verbatiminput{example2.1}


\section{Special Functions (scipy.special)}


\subsection{Vectorizing functions (scipy.special.GeneralFunction)}

One of the features that the \textbf{special} sub-package provides
is a class \textbf{GeneralFunction} to convert an ordinary Python
function which accepts scalars and returns scalars into a {}``vectorized-function{}''
with the same broadcasting rules as other Numeric functions (\emph{i.e.}
the Universal functions, or ufuncs). For example, suppose you have
a Python function named \textbf{addsubtract} defined as:

\verbatiminput{example3.1}which defines a function of two scalar
variables and returns a scalar result. The class GeneralFunction can
be used to {}``vectorize{}'' this function so that \begin{verbatim}
>>> vec_addsubstract = scipy.special.GeneralFunction(addsubtract) \end{verbatim} 
\noindent returns a function which takes array arguments and returns
an array result:

\verbatiminput{example3.2}


\subsection{Special Functions}

The main feature of the \textbf{special} package is the definition
of numerous special functions of mathematical physics. Available are
airy, elliptic, bessel, gamma, beta, hypergeometric, and several statistical
functions. All of these functions behave can take array arguments
and return array results following the same broadcasting rules as
other math functions in Numerical Python. For a complete list of these
functions with a one-line description type \texttt{>>>help(scipy.special).}
Each function also has it's own documentation accessible using help.


\section{Integration (scipy.integrate)}

The \textbf{integrate} sub-package provides several integration techniques
including an ordinary differential equation integrator. An overview
of the module is provided by the help command:

\verbatiminput{example4.1}


\subsection{General integration (scipy.integrate.quad)}

The function \textbf{quad} is provided to integrate a function of
one variable between two points. The points can be \( \pm \infty  \)
(\( \pm  \)scipy.integrate.Inf) to indicate infinite limits. For
example, suppose you wish to integrate a bessel function \texttt{jv(2.5,x)}
along the interval \( [0,4.5]. \) \[
I=\int _{0}^{4.5}J_{2.5}\left( x\right) \, dx.\]
 This could be computed using \textbf{quad:}

\verbatiminput{example4.2}

The first argument to quad is a {}``callable{}'' Python object (\emph{i.e}
a function, method, or class instance). Notice the use of a lambda-function
in this case as the argument. The next two arguments are the limits
of integration. The return value is a tuple, with the first element
holding the estimated value of the integral and the second element
holding an upper bound on the error. Notice, that in this case, the
true value of this integral is \[
I=\sqrt{\frac{2}{\pi }}\left( \frac{18}{27}\sqrt{2}\cos \left( 4.5\right) -\frac{4}{27}\sqrt{2}\sin \left( 4.5\right) +\sqrt{2\pi }\textrm{Si}\left( \frac{3}{\sqrt{\pi }}\right) \right) ,\]
 where \[
\textrm{Si}\left( x\right) =\int _{0}^{x}\sin \left( \frac{\pi }{2}t^{2}\right) \, dt.\]
 is the Fresnel sine integral. Note that the numerically-computed
integral is within \( 1.04\times 10^{-11} \) of the exact result
--- well below the reported error bound. 

Infinite inputs are also allowed in \textbf{quad} by using \( \pm  \)\textbf{scipy.integrate.Inf}
as one of the arguments. For example, suppose that a numerical value
for the exponential integral:\[
E_{n}\left( x\right) =\int _{1}^{\infty }\frac{e^{-xt}}{t^{n}}\, dt.\]
 is desired (and the fact that this integral can be computed as \texttt{scipy.special.expn(n,x)}
is forgotten). The functionality of the function \textbf{scipy.special.expn}
can be replicated by defining a new function \textbf{vec\_expint}
based on the routine \textbf{quad: }

\verbatiminput{example4.3} 

The function which is integrated can even use the quad argument (though
the error bound may underestimate the error due to possible numerical
error in the integrand from the use of \textbf{quad}. The integral
in this case is \[
I_{n}=\int _{0}^{\infty }\int _{1}^{\infty }\frac{e^{-xt}}{t^{n}}\, dt\, dx=\frac{1}{n}.\]


\verbatiminput{example4.4}

This last example shows that multiple integration can be handled using
repeated calls to \textbf{quad.} The mechanics of this for double
and triple integration have been wrapped up into the functions \textbf{dblquad}
and \textbf{tplquad.} The function, \textbf{dblquad} performs double
integration. Use the help function to be sure that the arguments are
defined in the correct order. In addition, the limits on all inner
integrals are actually functions which can be constant functions.
An example of using double integration to compute several values of
\( I_{n} \) is shown below:

\verbatiminput{example4.5}


\subsection{Ordinary differential equations (scipy.integrate.odeint)}
\end{document}
